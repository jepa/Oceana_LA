\documentclass[]{article}
\usepackage{lmodern}
\usepackage{amssymb,amsmath}
\usepackage{ifxetex,ifluatex}
\usepackage{fixltx2e} % provides \textsubscript
\ifnum 0\ifxetex 1\fi\ifluatex 1\fi=0 % if pdftex
  \usepackage[T1]{fontenc}
  \usepackage[utf8]{inputenc}
\else % if luatex or xelatex
  \ifxetex
    \usepackage{mathspec}
  \else
    \usepackage{fontspec}
  \fi
  \defaultfontfeatures{Ligatures=TeX,Scale=MatchLowercase}
\fi
% use upquote if available, for straight quotes in verbatim environments
\IfFileExists{upquote.sty}{\usepackage{upquote}}{}
% use microtype if available
\IfFileExists{microtype.sty}{%
\usepackage{microtype}
\UseMicrotypeSet[protrusion]{basicmath} % disable protrusion for tt fonts
}{}
\usepackage[margin=1in]{geometry}
\usepackage{hyperref}
\hypersetup{unicode=true,
            pdftitle={Response to Comments for the Chilean part of the Report},
            pdfborder={0 0 0},
            breaklinks=true}
\urlstyle{same}  % don't use monospace font for urls
\usepackage{longtable,booktabs}
\usepackage{graphicx,grffile}
\makeatletter
\def\maxwidth{\ifdim\Gin@nat@width>\linewidth\linewidth\else\Gin@nat@width\fi}
\def\maxheight{\ifdim\Gin@nat@height>\textheight\textheight\else\Gin@nat@height\fi}
\makeatother
% Scale images if necessary, so that they will not overflow the page
% margins by default, and it is still possible to overwrite the defaults
% using explicit options in \includegraphics[width, height, ...]{}
\setkeys{Gin}{width=\maxwidth,height=\maxheight,keepaspectratio}
\IfFileExists{parskip.sty}{%
\usepackage{parskip}
}{% else
\setlength{\parindent}{0pt}
\setlength{\parskip}{6pt plus 2pt minus 1pt}
}
\setlength{\emergencystretch}{3em}  % prevent overfull lines
\providecommand{\tightlist}{%
  \setlength{\itemsep}{0pt}\setlength{\parskip}{0pt}}
\setcounter{secnumdepth}{0}
% Redefines (sub)paragraphs to behave more like sections
\ifx\paragraph\undefined\else
\let\oldparagraph\paragraph
\renewcommand{\paragraph}[1]{\oldparagraph{#1}\mbox{}}
\fi
\ifx\subparagraph\undefined\else
\let\oldsubparagraph\subparagraph
\renewcommand{\subparagraph}[1]{\oldsubparagraph{#1}\mbox{}}
\fi

%%% Use protect on footnotes to avoid problems with footnotes in titles
\let\rmarkdownfootnote\footnote%
\def\footnote{\protect\rmarkdownfootnote}

%%% Change title format to be more compact
\usepackage{titling}

% Create subtitle command for use in maketitle
\newcommand{\subtitle}[1]{
  \posttitle{
    \begin{center}\large#1\end{center}
    }
}

\setlength{\droptitle}{-2em}

  \title{Response to Comments for the Chilean part of the Report}
    \pretitle{\vspace{\droptitle}\centering\huge}
  \posttitle{\par}
    \author{}
    \preauthor{}\postauthor{}
    \date{}
    \predate{}\postdate{}
  

\begin{document}
\maketitle

\section{Methods}\label{methods}

\begin{itemize}
\item
  Shouldn't it be ``minus E'', since the description of NDFS is: Net
  domestic fish supply. Total amount of fish produced in the country,
  minus exports and discards

  -- Yes!
\item
  I don't understand why ``A'' is being added within the parenthesis
  ``()'' and not outside of it We multiplied landings x Discards and
  Aquaculture production x production loss

  -- Should be \(NSS_i=[(L_i-L_i*O_{i})+A_i]*P_{i}-E_i+I_i\) -- Note
  that this is wrong in the report but not in the actual estimation
\end{itemize}

\section{Results}\label{results}

\begin{itemize}
\item
  This table states clearly that in Chile ``Captures'' are more
  important in terms of volume than Aquaculture, and then it is stated
  that with the wild caught species the most important one in Chile is
  anchoveta.

  -- According to the data
\item
  Regarding mote sculpin, I think it is wierd it's being included since
  from what I know, it's difficult to distinguish from sardines
\item
  Data
\item
  Why were these species selected? I think it would be good to include
  common hake for Chile, considering it is one of the species we have a
  campaign about.

  -- These are the species with higher landings in average between 2012
  and 2016.\\
  -- \emph{Southern hake} is level 11 with 1.14\% of Chile's total
  landings according to FAO's data.\\
  -- Note that according to FAO's data, 59.83\% of all chilean landings
  correspond for the first three groups.
\end{itemize}

\begin{longtable}[]{@{}llrr@{}}
\caption{Average of 2012-2016, tonnes.}\tabularnewline
\toprule
Country & MatchName & Mean\_Spp & Proportion\tabularnewline
\midrule
\endfirsthead
\toprule
Country & MatchName & Mean\_Spp & Proportion\tabularnewline
\midrule
\endhead
Chile & Anchoveta & 680,540.2 & 28.7\tabularnewline
Chile & Araucanian herring & 468,999.2 & 19.8\tabularnewline
Chile & Chilean jack mackerel & 268,498.2 & 11.3\tabularnewline
Chile & Chilean kelp & 214,622.4 & 9.1\tabularnewline
Chile & Jumbo flying squid & 150,487.2 & 6.3\tabularnewline
Chile & Mote sculpin & 43,343.2 & 1.8\tabularnewline
Chile & Patagonian grenadier & 42,941.0 & 1.8\tabularnewline
Chile & Pacific chub mackerel & 36,880.8 & 1.6\tabularnewline
Chile & Gracilaria seaweeds & 34,889.8 & 1.5\tabularnewline
Chile & Chilean sea urchin & 30,424.4 & 1.3\tabularnewline
Chile & South Pacific hake & 27,093.4 & 1.1\tabularnewline
Chile & Falkland sprat & 26,657.8 & 1.1\tabularnewline
Chile & Southern rays bream & 26,589.6 & 1.1\tabularnewline
\bottomrule
\end{longtable}

\begin{itemize}
\item
  Estimating National Seafood Supply, some highlights but no comments
\item
  Regarding mote sculpin, I think it is wierd it's being included since
  from what I know, it's difficult to distinguish from sardines

  -- Again, that is what the data says\ldots{}
\item
  Why are exports of species not considered in this table ? For example
  in Chile most salmon (Atlantic, coho and trout), about 85 percents it,
  is exported. So it wouldn't be truly contributing to our national food
  supply

  -- Exports are considered (Net trade = Imports-Exports), hence, the
  negative number for salmon in Chile. However, it is likely that we
  missed products since it is very hard to match all species and
  products in the FAO database
\end{itemize}

\begin{longtable}[]{@{}lrrr@{}}
\caption{Average of 2010-2016, tonnes.}\tabularnewline
\toprule
Species & T\_Imports & T\_Exports & Net\_Trade\tabularnewline
\midrule
\endfirsthead
\toprule
Species & T\_Imports & T\_Exports & Net\_Trade\tabularnewline
\midrule
\endhead
Atlantic salmon & 44.0 & 132,010.1 & -131,966.1\tabularnewline
Salmons, trouts & 13.3 & 100,784.4 & -100,771.1\tabularnewline
Salmonids & 109.8 & 74,392.5 & -74,282.7\tabularnewline
Sockeye salmon & 0.0 & 54.4 & -54.4\tabularnewline
Chinook salmon & 0.0 & 0.0 & 0.0\tabularnewline
Coho salmon & 0.0 & 0.0 & 0.0\tabularnewline
\bottomrule
\end{longtable}

\begin{itemize}
\tightlist
\item
  Salmon is included now for Chile, 3 times separately\\
  -- This is exactly our previous point:\\
  --- Atlantic salmon\\
  --- Salmons, trouts\\
  --- Salmonids
\end{itemize}


\end{document}
